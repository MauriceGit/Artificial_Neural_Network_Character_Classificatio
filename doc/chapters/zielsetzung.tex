\chapter{Thema und Zielsetzung}
\label{goal}
Die Aufgabenstellung fordert die Entwicklung eines neuronalen Netzes welches in der Lage ist die 26 Großbuchstaben (A..Z) des Alphabets zu erkennen. Ziel der Projektaufgabe ist, die Thematik der Neuronalen Netze mit Backpropagation Lernalgorithmus zu durchdringen, sich in ein Software-Framework zur Realisierung neuronaler Netze
einzuarbeiten und dessen Software-Komponenten zur Lösung der gestellten Aufgabe einzusetzen.

Die Anforderungen an das Framework sind wie folgt:
\begin{itemize}
\item Modifikation der Parameter: Lernrate, Momentum und Anfangsbelegung der Ge-
wichte
\item  Unterstützung für reine Feedforward-Netze
\item Unterstützung von Backpropagation
\item  Konstruktionen mehrlagiger Netze mit mindestens 1 und 2 versteckten Schichten
(hidden Layers) mit jeweils wählbarer Neuronenanzahlen
\item  Unterstützung für musterweises und epochenweises Lernen
\item  Identischer innerer Aufbau der Neuronen etwa mit sigmoider Aktivierungsfunktion
\end{itemize}

Anstelle ein bestehendes Framework zur Lösung der Aufgabe einzusetzen fiel die Wahl auf die Entwicklung eines eigenen neuronalen Netzes in Python. Die Zielsetzung war dabei, ein einfaches, leistungsfähiges Netzwerk zu entwickeln welches in der Lage ist die Aufgabe möglichst optimal zu bewältigen. Im Gegensatz zum Einsatz von größeren Frameworks wie \emph{scikit-learn} oder vergleichbaren Frameworks entsteht so kein Funktionsoverhead und es wird die Thematik grundlegend selbst erarbeitet.